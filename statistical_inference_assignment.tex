\documentclass[]{article}
\usepackage{lmodern}
\usepackage{amssymb,amsmath}
\usepackage{ifxetex,ifluatex}
\usepackage{fixltx2e} % provides \textsubscript
\ifnum 0\ifxetex 1\fi\ifluatex 1\fi=0 % if pdftex
  \usepackage[T1]{fontenc}
  \usepackage[utf8]{inputenc}
\else % if luatex or xelatex
  \ifxetex
    \usepackage{mathspec}
  \else
    \usepackage{fontspec}
  \fi
  \defaultfontfeatures{Ligatures=TeX,Scale=MatchLowercase}
\fi
% use upquote if available, for straight quotes in verbatim environments
\IfFileExists{upquote.sty}{\usepackage{upquote}}{}
% use microtype if available
\IfFileExists{microtype.sty}{%
\usepackage{microtype}
\UseMicrotypeSet[protrusion]{basicmath} % disable protrusion for tt fonts
}{}
\usepackage[margin=1in]{geometry}
\usepackage{hyperref}
\hypersetup{unicode=true,
            pdftitle={Statistical Inference},
            pdfauthor={Kaveri Chatra},
            pdfborder={0 0 0},
            breaklinks=true}
\urlstyle{same}  % don't use monospace font for urls
\usepackage{color}
\usepackage{fancyvrb}
\newcommand{\VerbBar}{|}
\newcommand{\VERB}{\Verb[commandchars=\\\{\}]}
\DefineVerbatimEnvironment{Highlighting}{Verbatim}{commandchars=\\\{\}}
% Add ',fontsize=\small' for more characters per line
\usepackage{framed}
\definecolor{shadecolor}{RGB}{248,248,248}
\newenvironment{Shaded}{\begin{snugshade}}{\end{snugshade}}
\newcommand{\KeywordTok}[1]{\textcolor[rgb]{0.13,0.29,0.53}{\textbf{#1}}}
\newcommand{\DataTypeTok}[1]{\textcolor[rgb]{0.13,0.29,0.53}{#1}}
\newcommand{\DecValTok}[1]{\textcolor[rgb]{0.00,0.00,0.81}{#1}}
\newcommand{\BaseNTok}[1]{\textcolor[rgb]{0.00,0.00,0.81}{#1}}
\newcommand{\FloatTok}[1]{\textcolor[rgb]{0.00,0.00,0.81}{#1}}
\newcommand{\ConstantTok}[1]{\textcolor[rgb]{0.00,0.00,0.00}{#1}}
\newcommand{\CharTok}[1]{\textcolor[rgb]{0.31,0.60,0.02}{#1}}
\newcommand{\SpecialCharTok}[1]{\textcolor[rgb]{0.00,0.00,0.00}{#1}}
\newcommand{\StringTok}[1]{\textcolor[rgb]{0.31,0.60,0.02}{#1}}
\newcommand{\VerbatimStringTok}[1]{\textcolor[rgb]{0.31,0.60,0.02}{#1}}
\newcommand{\SpecialStringTok}[1]{\textcolor[rgb]{0.31,0.60,0.02}{#1}}
\newcommand{\ImportTok}[1]{#1}
\newcommand{\CommentTok}[1]{\textcolor[rgb]{0.56,0.35,0.01}{\textit{#1}}}
\newcommand{\DocumentationTok}[1]{\textcolor[rgb]{0.56,0.35,0.01}{\textbf{\textit{#1}}}}
\newcommand{\AnnotationTok}[1]{\textcolor[rgb]{0.56,0.35,0.01}{\textbf{\textit{#1}}}}
\newcommand{\CommentVarTok}[1]{\textcolor[rgb]{0.56,0.35,0.01}{\textbf{\textit{#1}}}}
\newcommand{\OtherTok}[1]{\textcolor[rgb]{0.56,0.35,0.01}{#1}}
\newcommand{\FunctionTok}[1]{\textcolor[rgb]{0.00,0.00,0.00}{#1}}
\newcommand{\VariableTok}[1]{\textcolor[rgb]{0.00,0.00,0.00}{#1}}
\newcommand{\ControlFlowTok}[1]{\textcolor[rgb]{0.13,0.29,0.53}{\textbf{#1}}}
\newcommand{\OperatorTok}[1]{\textcolor[rgb]{0.81,0.36,0.00}{\textbf{#1}}}
\newcommand{\BuiltInTok}[1]{#1}
\newcommand{\ExtensionTok}[1]{#1}
\newcommand{\PreprocessorTok}[1]{\textcolor[rgb]{0.56,0.35,0.01}{\textit{#1}}}
\newcommand{\AttributeTok}[1]{\textcolor[rgb]{0.77,0.63,0.00}{#1}}
\newcommand{\RegionMarkerTok}[1]{#1}
\newcommand{\InformationTok}[1]{\textcolor[rgb]{0.56,0.35,0.01}{\textbf{\textit{#1}}}}
\newcommand{\WarningTok}[1]{\textcolor[rgb]{0.56,0.35,0.01}{\textbf{\textit{#1}}}}
\newcommand{\AlertTok}[1]{\textcolor[rgb]{0.94,0.16,0.16}{#1}}
\newcommand{\ErrorTok}[1]{\textcolor[rgb]{0.64,0.00,0.00}{\textbf{#1}}}
\newcommand{\NormalTok}[1]{#1}
\usepackage{graphicx,grffile}
\makeatletter
\def\maxwidth{\ifdim\Gin@nat@width>\linewidth\linewidth\else\Gin@nat@width\fi}
\def\maxheight{\ifdim\Gin@nat@height>\textheight\textheight\else\Gin@nat@height\fi}
\makeatother
% Scale images if necessary, so that they will not overflow the page
% margins by default, and it is still possible to overwrite the defaults
% using explicit options in \includegraphics[width, height, ...]{}
\setkeys{Gin}{width=\maxwidth,height=\maxheight,keepaspectratio}
\IfFileExists{parskip.sty}{%
\usepackage{parskip}
}{% else
\setlength{\parindent}{0pt}
\setlength{\parskip}{6pt plus 2pt minus 1pt}
}
\setlength{\emergencystretch}{3em}  % prevent overfull lines
\providecommand{\tightlist}{%
  \setlength{\itemsep}{0pt}\setlength{\parskip}{0pt}}
\setcounter{secnumdepth}{0}
% Redefines (sub)paragraphs to behave more like sections
\ifx\paragraph\undefined\else
\let\oldparagraph\paragraph
\renewcommand{\paragraph}[1]{\oldparagraph{#1}\mbox{}}
\fi
\ifx\subparagraph\undefined\else
\let\oldsubparagraph\subparagraph
\renewcommand{\subparagraph}[1]{\oldsubparagraph{#1}\mbox{}}
\fi

%%% Use protect on footnotes to avoid problems with footnotes in titles
\let\rmarkdownfootnote\footnote%
\def\footnote{\protect\rmarkdownfootnote}

%%% Change title format to be more compact
\usepackage{titling}

% Create subtitle command for use in maketitle
\newcommand{\subtitle}[1]{
  \posttitle{
    \begin{center}\large#1\end{center}
    }
}

\setlength{\droptitle}{-2em}

  \title{Statistical Inference}
    \pretitle{\vspace{\droptitle}\centering\huge}
  \posttitle{\par}
    \author{Kaveri Chatra}
    \preauthor{\centering\large\emph}
  \postauthor{\par}
    \date{}
    \predate{}\postdate{}
  

\begin{document}
\maketitle

\subsection{Part 1: Investigating exponetial distribution and comparing
with Central Limit
Theorem}\label{part-1-investigating-exponetial-distribution-and-comparing-with-central-limit-theorem}

\subsubsection{Loading Libraries}\label{loading-libraries}

\begin{Shaded}
\begin{Highlighting}[]
\KeywordTok{library}\NormalTok{(}\StringTok{"data.table"}\NormalTok{)}
\KeywordTok{library}\NormalTok{(}\StringTok{"ggplot2"}\NormalTok{)}
\KeywordTok{library}\NormalTok{(}\StringTok{"datasets"}\NormalTok{)}
\KeywordTok{library}\NormalTok{(}\StringTok{"rcompanion"}\NormalTok{)}
\end{Highlighting}
\end{Shaded}

\begin{Shaded}
\begin{Highlighting}[]
\KeywordTok{set.seed}\NormalTok{(}\DecValTok{1}\NormalTok{)}
\NormalTok{lambda <-}\StringTok{ }\FloatTok{0.2}
\NormalTok{sim<-}\StringTok{ }\DecValTok{1000}
\NormalTok{n <-}\StringTok{ }\DecValTok{40}


\NormalTok{exp_sim <-}\StringTok{ }\KeywordTok{replicate}\NormalTok{(sim, }\KeywordTok{rexp}\NormalTok{(n, lambda))}
\NormalTok{means_exp <-}\StringTok{ }\KeywordTok{apply}\NormalTok{(exp_sim, }\DecValTok{2}\NormalTok{, mean)}
\end{Highlighting}
\end{Shaded}

\subsubsection{1. Show where the distribution is centered at and compare
it to the theoretical center of the
distribution.}\label{show-where-the-distribution-is-centered-at-and-compare-it-to-the-theoretical-center-of-the-distribution.}

\begin{Shaded}
\begin{Highlighting}[]
\NormalTok{sample_mean <-}\StringTok{ }\KeywordTok{mean}\NormalTok{(exp_sim)}
\NormalTok{theoretical_mean <-}\DecValTok{1}\OperatorTok{/}\NormalTok{lambda}
\KeywordTok{hist}\NormalTok{(means_exp, }\DataTypeTok{xlab =} \StringTok{"sample means"}\NormalTok{, }\DataTypeTok{main =} \StringTok{"Exponential Function"}\NormalTok{)}
\KeywordTok{abline}\NormalTok{(}\DataTypeTok{v =}\NormalTok{ theoretical_mean, }\DataTypeTok{col =} \StringTok{"orange"}\NormalTok{, }\DataTypeTok{lwd =} \DecValTok{4}\NormalTok{)}
\KeywordTok{abline}\NormalTok{(}\DataTypeTok{v =}\NormalTok{ sample_mean, }\DataTypeTok{col =} \StringTok{"black"}\NormalTok{)}
\end{Highlighting}
\end{Shaded}

\includegraphics{statistical_inference_assignment_files/figure-latex/unnamed-chunk-3-1.pdf}
From the graph it can be noticed that sample mean is very close to
theoretical mean

\subsubsection{2. Show how variable it is and compare it to the
theoretical variance of the
distribution..}\label{show-how-variable-it-is-and-compare-it-to-the-theoretical-variance-of-the-distribution..}

Standard deviation

\begin{Shaded}
\begin{Highlighting}[]
\NormalTok{sample_std <-}\StringTok{ }\KeywordTok{sd}\NormalTok{(means_exp)}
\CommentTok{#For exp. distribution mean = std}
\NormalTok{theoretical_std <-}\StringTok{ }\DecValTok{1}\OperatorTok{/}\NormalTok{lambda}\OperatorTok{/}\KeywordTok{sqrt}\NormalTok{(n)}
\NormalTok{theoretical_std}
\end{Highlighting}
\end{Shaded}

\begin{verbatim}
## [1] 0.7905694
\end{verbatim}

\begin{Shaded}
\begin{Highlighting}[]
\NormalTok{sample_std}
\end{Highlighting}
\end{Shaded}

\begin{verbatim}
## [1] 0.7817394
\end{verbatim}

Variance

\begin{Shaded}
\begin{Highlighting}[]
\NormalTok{sample_var <-}\StringTok{ }\KeywordTok{var}\NormalTok{(means_exp)}
\NormalTok{theoretical_var <-}\StringTok{ }\DecValTok{1}\OperatorTok{/}\NormalTok{lambda}\OperatorTok{^}\DecValTok{2}\OperatorTok{/}\NormalTok{n}
\NormalTok{theoretical_var }
\end{Highlighting}
\end{Shaded}

\begin{verbatim}
## [1] 0.625
\end{verbatim}

\begin{Shaded}
\begin{Highlighting}[]
\NormalTok{sample_var}
\end{Highlighting}
\end{Shaded}

\begin{verbatim}
## [1] 0.6111165
\end{verbatim}

\subsubsection{3. Show that the distribution is approximately
normal.}\label{show-that-the-distribution-is-approximately-normal.}

From the Central Limit Theorem, the distribution of averages is often
normal, even if the distribution that the data is being sampled from is
non-normal.

\begin{Shaded}
\begin{Highlighting}[]
\KeywordTok{par}\NormalTok{(}\DataTypeTok{mfrow =} \KeywordTok{c}\NormalTok{(}\DecValTok{1}\NormalTok{,}\DecValTok{1}\NormalTok{))}
\KeywordTok{hist}\NormalTok{(means_exp,}\DataTypeTok{breaks=}\NormalTok{n,}\DataTypeTok{prob=}\NormalTok{T,}\DataTypeTok{xlab =} \StringTok{"means"}\NormalTok{,}\DataTypeTok{ylab=}\StringTok{"density"}\NormalTok{)}
\end{Highlighting}
\end{Shaded}

\includegraphics{statistical_inference_assignment_files/figure-latex/unnamed-chunk-6-1.pdf}

\begin{Shaded}
\begin{Highlighting}[]
\CommentTok{#curve(dnorm(x, 0, 1), -3, 3, col = 'blue',add=T) }
\CommentTok{#lines(density(scale(means_exp)), col = 'red')}
\end{Highlighting}
\end{Shaded}

\subsection{Part 2: Basic Inferential Data
Analysis}\label{part-2-basic-inferential-data-analysis}

\subsubsection{Loading and performing some exploratory data
analysis}\label{loading-and-performing-some-exploratory-data-analysis}

\begin{Shaded}
\begin{Highlighting}[]
\KeywordTok{head}\NormalTok{(ToothGrowth)}
\end{Highlighting}
\end{Shaded}

\begin{verbatim}
##    len supp dose
## 1  4.2   VC  0.5
## 2 11.5   VC  0.5
## 3  7.3   VC  0.5
## 4  5.8   VC  0.5
## 5  6.4   VC  0.5
## 6 10.0   VC  0.5
\end{verbatim}

\begin{Shaded}
\begin{Highlighting}[]
\KeywordTok{dim}\NormalTok{(ToothGrowth)}
\end{Highlighting}
\end{Shaded}

\begin{verbatim}
## [1] 60  3
\end{verbatim}

\begin{Shaded}
\begin{Highlighting}[]
\KeywordTok{str}\NormalTok{(ToothGrowth)}
\end{Highlighting}
\end{Shaded}

\begin{verbatim}
## 'data.frame':    60 obs. of  3 variables:
##  $ len : num  4.2 11.5 7.3 5.8 6.4 10 11.2 11.2 5.2 7 ...
##  $ supp: Factor w/ 2 levels "OJ","VC": 2 2 2 2 2 2 2 2 2 2 ...
##  $ dose: num  0.5 0.5 0.5 0.5 0.5 0.5 0.5 0.5 0.5 0.5 ...
\end{verbatim}

\begin{Shaded}
\begin{Highlighting}[]
\KeywordTok{summary}\NormalTok{(ToothGrowth)}
\end{Highlighting}
\end{Shaded}

\begin{verbatim}
##       len        supp         dose      
##  Min.   : 4.20   OJ:30   Min.   :0.500  
##  1st Qu.:13.07   VC:30   1st Qu.:0.500  
##  Median :19.25           Median :1.000  
##  Mean   :18.81           Mean   :1.167  
##  3rd Qu.:25.27           3rd Qu.:2.000  
##  Max.   :33.90           Max.   :2.000
\end{verbatim}

\begin{Shaded}
\begin{Highlighting}[]
\KeywordTok{unique}\NormalTok{(ToothGrowth}\OperatorTok{$}\NormalTok{dose)}
\end{Highlighting}
\end{Shaded}

\begin{verbatim}
## [1] 0.5 1.0 2.0
\end{verbatim}

\paragraph{Plots}\label{plots}

\begin{Shaded}
\begin{Highlighting}[]
\NormalTok{g <-}\StringTok{ }\KeywordTok{ggplot}\NormalTok{(ToothGrowth, }\KeywordTok{aes}\NormalTok{(len, supp, }\DataTypeTok{color =}\NormalTok{ dose)) }\OperatorTok{+}\StringTok{ }\KeywordTok{geom_point}\NormalTok{()}
\KeywordTok{print}\NormalTok{(g)}
\end{Highlighting}
\end{Shaded}

\includegraphics{statistical_inference_assignment_files/figure-latex/unnamed-chunk-8-1.pdf}

\begin{Shaded}
\begin{Highlighting}[]
\NormalTok{g <-}\StringTok{ }\KeywordTok{ggplot}\NormalTok{(ToothGrowth, }\KeywordTok{aes}\NormalTok{(}\KeywordTok{factor}\NormalTok{(dose), len, }\DataTypeTok{color =}\NormalTok{ supp)) }\OperatorTok{+}\StringTok{ }\KeywordTok{geom_boxplot}\NormalTok{() }\OperatorTok{+}\StringTok{ }\KeywordTok{facet_grid}\NormalTok{(.}\OperatorTok{~}\NormalTok{dose)}
\KeywordTok{print}\NormalTok{(g)}
\end{Highlighting}
\end{Shaded}

\includegraphics{statistical_inference_assignment_files/figure-latex/unnamed-chunk-9-1.pdf}

\subsubsection{Using confidence intervals and/or hypothesis tests to
compare tooth growth by supp and
dose}\label{using-confidence-intervals-andor-hypothesis-tests-to-compare-tooth-growth-by-supp-and-dose}

\paragraph{Confidence interval}\label{confidence-interval}

\begin{Shaded}
\begin{Highlighting}[]
\KeywordTok{groupwiseMean}\NormalTok{(len }\OperatorTok{~}\StringTok{ }\NormalTok{supp, }\DataTypeTok{data =}\NormalTok{ ToothGrowth, }\DataTypeTok{conf =} \FloatTok{0.95}\NormalTok{, }\DataTypeTok{digits =} \DecValTok{3}\NormalTok{)}
\end{Highlighting}
\end{Shaded}

\begin{verbatim}
##   supp  n Mean Conf.level Trad.lower Trad.upper
## 1   OJ 30 20.7       0.95       18.2       23.1
## 2   VC 30 17.0       0.95       13.9       20.0
\end{verbatim}

\begin{Shaded}
\begin{Highlighting}[]
\KeywordTok{groupwiseMean}\NormalTok{(len }\OperatorTok{~}\StringTok{ }\KeywordTok{factor}\NormalTok{(dose), }\DataTypeTok{data =}\NormalTok{ ToothGrowth, }\DataTypeTok{conf =} \FloatTok{0.95}\NormalTok{, }\DataTypeTok{digits =} \DecValTok{3}\NormalTok{)}
\end{Highlighting}
\end{Shaded}

\begin{verbatim}
##   dose  n Mean Conf.level Trad.lower Trad.upper
## 1  0.5 20 10.6       0.95        8.5       12.7
## 2  1.0 20 19.7       0.95       17.7       21.8
## 3  2.0 20 26.1       0.95       24.3       27.9
\end{verbatim}

\begin{Shaded}
\begin{Highlighting}[]
\KeywordTok{groupwiseMean}\NormalTok{(len }\OperatorTok{~}\StringTok{ }\KeywordTok{factor}\NormalTok{(dose) }\OperatorTok{+}\StringTok{ }\KeywordTok{factor}\NormalTok{(supp), }\DataTypeTok{data =}\NormalTok{ ToothGrowth, }\DataTypeTok{conf =} \FloatTok{0.95}\NormalTok{, }\DataTypeTok{digits =} \DecValTok{3}\NormalTok{)}
\end{Highlighting}
\end{Shaded}

\begin{verbatim}
##   dose supp  n  Mean Conf.level Trad.lower Trad.upper
## 1  0.5   OJ 10 13.20       0.95      10.00      16.40
## 2  0.5   VC 10  7.98       0.95       6.02       9.94
## 3  1.0   OJ 10 22.70       0.95      19.90      25.50
## 4  1.0   VC 10 16.80       0.95      15.00      18.60
## 5  2.0   OJ 10 26.10       0.95      24.20      28.00
## 6  2.0   VC 10 26.10       0.95      22.70      29.60
\end{verbatim}

\paragraph{T test}\label{t-test}

Tooth length corresponding to OJ vs VC supplement. Null hypothesis being
means of both are same and Alternate hypothesis being mean of tooth
growth corresponding to OJ is greater than that of VC supplements.

\begin{Shaded}
\begin{Highlighting}[]
\NormalTok{OJ <-}\StringTok{ }\NormalTok{ToothGrowth}\OperatorTok{$}\NormalTok{len[ToothGrowth}\OperatorTok{$}\NormalTok{supp }\OperatorTok{==}\StringTok{ 'OJ'}\NormalTok{]}
\NormalTok{VC =}\StringTok{ }\NormalTok{ToothGrowth}\OperatorTok{$}\NormalTok{len[ToothGrowth}\OperatorTok{$}\NormalTok{supp }\OperatorTok{==}\StringTok{ 'VC'}\NormalTok{]}
\KeywordTok{t.test}\NormalTok{(OJ, VC, }\DataTypeTok{alternative =} \StringTok{"greater"}\NormalTok{, }\DataTypeTok{paired =} \OtherTok{FALSE}\NormalTok{, }\DataTypeTok{var.equal =} \OtherTok{FALSE}\NormalTok{, }\DataTypeTok{conf.level =} \FloatTok{0.95}\NormalTok{)}
\end{Highlighting}
\end{Shaded}

\begin{verbatim}
## 
##  Welch Two Sample t-test
## 
## data:  OJ and VC
## t = 1.9153, df = 55.309, p-value = 0.03032
## alternative hypothesis: true difference in means is greater than 0
## 95 percent confidence interval:
##  0.4682687       Inf
## sample estimates:
## mean of x mean of y 
##  20.66333  16.96333
\end{verbatim}

Tooth Growth by dose. Dose is divided into 3 groups, group1 = 0.5,
group2 = 1.0, group3 = 2.0

\begin{Shaded}
\begin{Highlighting}[]
\NormalTok{group1 <-}\StringTok{ }\NormalTok{ToothGrowth}\OperatorTok{$}\NormalTok{len[ToothGrowth}\OperatorTok{$}\NormalTok{dose }\OperatorTok{==}\StringTok{ }\FloatTok{0.5}\NormalTok{]}
\NormalTok{group2 <-}\StringTok{ }\NormalTok{ToothGrowth}\OperatorTok{$}\NormalTok{len[ToothGrowth}\OperatorTok{$}\NormalTok{dose }\OperatorTok{==}\StringTok{ }\FloatTok{1.0}\NormalTok{]}
\NormalTok{group3 <-}\StringTok{ }\NormalTok{ToothGrowth}\OperatorTok{$}\NormalTok{len[ToothGrowth}\OperatorTok{$}\NormalTok{dose }\OperatorTok{==}\StringTok{ }\FloatTok{2.0}\NormalTok{]}

\KeywordTok{t.test}\NormalTok{(group1, group2, }\DataTypeTok{alternative =} \StringTok{"less"}\NormalTok{, }\DataTypeTok{paired =} \OtherTok{FALSE}\NormalTok{, }\DataTypeTok{var.equal =} \OtherTok{FALSE}\NormalTok{, }\DataTypeTok{conf.level =} \FloatTok{0.95}\NormalTok{)}
\end{Highlighting}
\end{Shaded}

\begin{verbatim}
## 
##  Welch Two Sample t-test
## 
## data:  group1 and group2
## t = -6.4766, df = 37.986, p-value = 6.342e-08
## alternative hypothesis: true difference in means is less than 0
## 95 percent confidence interval:
##       -Inf -6.753323
## sample estimates:
## mean of x mean of y 
##    10.605    19.735
\end{verbatim}

\begin{Shaded}
\begin{Highlighting}[]
\KeywordTok{t.test}\NormalTok{(group2, group3, }\DataTypeTok{alternative =} \StringTok{"less"}\NormalTok{, }\DataTypeTok{paired =} \OtherTok{FALSE}\NormalTok{, }\DataTypeTok{var.equal =} \OtherTok{FALSE}\NormalTok{, }\DataTypeTok{conf.level =} \FloatTok{0.95}\NormalTok{)}
\end{Highlighting}
\end{Shaded}

\begin{verbatim}
## 
##  Welch Two Sample t-test
## 
## data:  group2 and group3
## t = -4.9005, df = 37.101, p-value = 9.532e-06
## alternative hypothesis: true difference in means is less than 0
## 95 percent confidence interval:
##      -Inf -4.17387
## sample estimates:
## mean of x mean of y 
##    19.735    26.100
\end{verbatim}

\begin{Shaded}
\begin{Highlighting}[]
\NormalTok{OJtwomg <-}\StringTok{ }\NormalTok{ToothGrowth}\OperatorTok{$}\NormalTok{len[ToothGrowth}\OperatorTok{$}\NormalTok{supp }\OperatorTok{==}\StringTok{ 'OJ'} \OperatorTok{&}\StringTok{ }\NormalTok{ToothGrowth}\OperatorTok{$}\NormalTok{dose }\OperatorTok{==}\StringTok{ }\FloatTok{2.0}\NormalTok{]}
\NormalTok{VCtwomg <-}\StringTok{ }\NormalTok{ToothGrowth}\OperatorTok{$}\NormalTok{len[ToothGrowth}\OperatorTok{$}\NormalTok{supp }\OperatorTok{==}\StringTok{ 'VC'} \OperatorTok{&}\StringTok{ }\NormalTok{ToothGrowth}\OperatorTok{$}\NormalTok{dose }\OperatorTok{==}\StringTok{ }\FloatTok{2.0}\NormalTok{]}

\KeywordTok{t.test}\NormalTok{(OJtwomg, VCtwomg, }\DataTypeTok{alternative =} \StringTok{"two.sided"}\NormalTok{, }\DataTypeTok{paired =} \OtherTok{FALSE}\NormalTok{, }\DataTypeTok{var.equal =} \OtherTok{FALSE}\NormalTok{, }\DataTypeTok{conf.level =} \FloatTok{0.95}\NormalTok{)}
\end{Highlighting}
\end{Shaded}

\begin{verbatim}
## 
##  Welch Two Sample t-test
## 
## data:  OJtwomg and VCtwomg
## t = -0.046136, df = 14.04, p-value = 0.9639
## alternative hypothesis: true difference in means is not equal to 0
## 95 percent confidence interval:
##  -3.79807  3.63807
## sample estimates:
## mean of x mean of y 
##     26.06     26.14
\end{verbatim}

\subsubsection{Conclusions}\label{conclusions}

From the result of the T tests and confidence intervals, it can be
concluded that:

\begin{verbatim}
1.  The supplement OJ had more impact than supplement VC on tooth growth.
2.  Higher the dose more is the growth.
3.  The impact on tooth growth 2.0 mg dose with VC supplement vs 2.0 mg dose with OJ supplement can't be determined.
\end{verbatim}


\end{document}
